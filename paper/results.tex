\section{Experimental Results}
\label{sec:results}


To test this hypothesis, we use two modern SAT solvers, CryptoMiniSat and Glucose, and an older SAT solver MiniSat. These solvers are described in more detail in Section \ref{sec:related:solvers}. 
We hypothesize that both CryptoMiniSat and Glucose will outperform MiniSat for these attacks. By comparing modern solvers to a contemporary solver, instead of comparing the performance of modern solvers directly to the results reported in each prior work, we hope to isolate the algorithmic improvement of SAT solvers from the improvement in the underlying hardware.


\begin{table}[!htbp]
	\centering
	\begin{tabular}{|c|c|c|c|c|c|}
		\hline
		\textbf{Solver} & \textbf{Parameters} & \textbf{Crypto1} & \textbf{MD4} & \textbf{DES} & \textbf{SHA-3} \\
		\hline
		MiniSat & - & 201.31 & 145.19 & 48.63 & .\\
		\hline
		\multirow{3}{*}{Glucose-Syrup} & 1 core & 157.76 & 1504.96 & 94.16 & .\\ \cline{2-6}
		& 4 cores & 64.53 & 499.33 & 54.30 & .\\ \cline{2-6}
		& 8 cores & 39.05 & 313.95 & 40.73 & .\\
		\hline
		
		\multirow{3}{*}{CryptoMiniSat} & v2 with Gaussian & 51.24 & 769.03 & 69.34 & . \\ \cline{2-6}
		& v4 with Gaussian & 53.62 & 578.59 & 50.87 & . \\ \cline{2-6}
		& v4 without Gaussian & 112.22 & 538.81 & 95.18 & .\\
		\hline
	\end{tabular}
	
	\caption{Median runtime on Cryptographic Benchmarks (s)}
	\label{table:crypto1:runtime}
\end{table}