\appendix
\section{Project Milestones}
\label{sec:milestones}

In the development of this project, we covered three major milestones. Note that we added Jonathan as a group member shortly after the initial project proposal and redistributed the work at that time.
\begin{enumerate}
	\item By March 16th, we prepared our toolset for the analysis. In particular, we set up the three SAT solvers listed in Section \ref{sec:related:solvers} on the Rice DAVinCI cluster.
	
	\item By March 30th, we completed the attack on Crypto-1 by Soos et. al \cite{SNC09}. We will work together on this task, each finding optimal parameters for a single solver. Jeffrey handled MiniSat, Jonathan handled Glucose, and Corey handled CryptoMiniSat.
	
	\item By April 22nd, we completed two other attacks. Jeffrey replicated the attack on MD4 by De at. al \cite{DKV07}, and Corey replicated the attack on DES by Courtois and Bard \cite{CB07}. Jonathan attempted to replicate the attack on SHA-3 by Morawiecki and Srebrny \cite{MS13}, but he was unable to reproduce their results.
\end{enumerate}

\section{SHA-3}
For their attack on Keccak, Morawiecki and Srebrny developed a tool called CryptLogVer to generate a CNF formula based on the Keccak hash function. We had planned to utilize this tool to reproduce the preimage attack. Unfortunately, the developers of CryptLogVer no longer maintain it, and it is unavailable for download. As the authors noted in their paper, alternatives to CryptLogVer, such as KeccakTools, do not generate CNF directly. Jonathan attempted to generate comparable CNF formulas based on KeccakTools' equation generation facility. However, the resulting transformation proved to be fragile and unconducive for use in reproducing the attack.