\section{Attacking DES}
The Data Encryption Standard (DES) is a symmetric-key block cipher algorithm based on Feistel networks. It was the NBS standard for cryptography between 1977 and the early 2000s, when brute force attacks on its 56-bit key started to become feasible.\cite{find appropriate citation}

DES normally takes place in 16 rounds. Bard and Cortois\cite{citation} used MiniSat2 to solve 6 rounds of DES in 68 seconds. We used their ANF files, available on Courtois's site\footnote{link}. We converted these ANF files into CNF files using Soos and Bettale's anf2cnf tool\footnote{link}. We found that even MiniSat2 could solve 6 rounds in under a second on our hardware. This is likely because they originally used a desktop machine. We instead solved other ANFs they had provided for attacking 8 rounds of DES with 16 chosen plaintexts to provide a more difficult problem to compare solvers on.

We found that 

%\begin{table}[!htbp]
%	\centering
%	\begin{tabular}{|c|c|c|}
%		\hline
%		\textbf{Solver} & \textbf{Parameters} & \textbf{Time (s)} \\
%		\hline
%		MiniSat & - & 189.25 \\
%		\hline
%		\multirow{3}{*}{Glucose-Syrup} & 1 core & 133.59 \\ \cline{2-3}
%		& 4 cores & 41.94 \\ \cline{2-3}
%		& 8 cores & 33.26 \\
%		\hline
%		
%		\multirow{3}{*}{CryptoMiniSat} & No {XORs} & 127.948 \\ \cline{2-3} 
%		& With {XORs} & 128.129 \\ \cline{2-3} 
%		& With {XORs} and Gaussian & 61.637 \\ \cline{2-3}
%		\hline
%	\end{tabular}
%	
%	\caption{Average runtime to break the Crypto-1 cipher with 56 bits of keystream}
%	\label{table:crypto1:runtime}
%\end{table}