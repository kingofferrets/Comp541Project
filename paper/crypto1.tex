\subsection{The Crypto-1 Stream-Cipher}
The Crypto-1 stream cipher is a simple cipher designed to have a minimal hardware footprint. We were able to solve the Crypto-1 cipher with 56 bits of known keystream. In 2009, Soos et. al \cite{SNC09} introduced the idea of XOR constraints on SAT solving, which led to the development of CryptoMiniSat. They attacked Bivium, Crypto-1, and HiTag2, with best success against Crypto-1. We generated 100 versions of the Crypto-1 cipher with Grain of Salt \footnote{http://www.msoos.org/grain-of-salt/}, a tool which can generate SAT formulas for a variety of shift register-based stream ciphers. We ran each solver on all 100 formulas and reported the average run time in Table \ref{table:crypto1:runtime}.

Compared to Soos et. al \cite{SNC09}, we observe a 2.6x speedup from hardware improvements alone. Moreover, Glucose was able to achieve a 1.4x speedup relative to the MiniSat. CryptoMiniSat performed similarly to Glucose without Gaussian elimination of XOR clauses. With Gaussian elimination, we achieved another 2x speedup. 

\begin{table}[!htbp]
	\centering
	\begin{tabular}{|c|c|c|}
		\hline
		\textbf{Solver} & \textbf{Parameters} & \textbf{Time (s)} \\
		\hline
		MiniSat & - & 221.54 \\
		\hline
		\multirow{3}{*}{Glucose-Syrup} & 1 core & 191.83 \\ \cline{2-3}
		& 4 cores & 68.86 \\ \cline{2-3}
		& 8 cores & 39.56 \\
		\hline
		
		\multirow{3}{*}{CryptoMiniSat} & v2.9 with Gaussian & 56.44 \\ \cline{2-3}
		& v4 with Gaussian & 61.50 \\ \cline{2-3}
		& v4 without Gaussian & 122.08 \\
		\hline
	\end{tabular}
	
	\caption{Average runtime to break the Crypto-1 cipher with 56 bits of keystream}
	\label{table:crypto1:runtime}
\end{table}