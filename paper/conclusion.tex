\section{Conclusion}
\label{sec:conclusion}
We developed SAT formulas that encode inversion attacks on MD4 and that encode key deduction attacks on the Crypto-1 Stream-Cipher and on DES. We ran these SAT formulas on a variety of older and modern SAT solvers. We found that the change in performance between older and modern solvers was highly dependent on the particular cryptographic problem.

The ``SAT revolution'' has driven heavy improvement in other areas, but this benefit has not reached the field of algebraic cryptanalysis. While hardware improvements over attacks from 10 years ago have been significant, it's unclear whether modern SAT solvers are generally better at all on cryptographic problems than those from 10 years ago. Instead, different SAT solvers using different heuristics are better or worse at different cryptanalytic problems.

These mixed results suggest trying a variety of SAT solvers when running cryptanalytic experiments. In all experiments we ran, the relative performance of SAT solvers was consistent across the set of benchmarks originating from the same problem; this suggests that running all SAT solvers in parallel on each instance of a problem is not the correct approach. Instead, future attempts at cryptanalysis using SAT solving should focus on finding particular SAT solvers and configurations that are effective on the problems they face. There is also an opportunity in the SAT solving community to find new heuristics and deduction rules designed for cryptographic attacks.