\section{Introduction}

Algebraic structure is often beneficial when developing cryptographic primitives. For example, Anshel, Anshel, and Goldfeld's public key algorithm \cite{AAG99} relies heavily on the algebraic properties of the cryptography. However, these algebraic properties can leave room for information to be deduced by the attacker without access to the key. Algebraic Cryptanalysis \cite{Bard09} is the study of breaking cryptography by solving algebraic equations which take advantage of underlying algebraic structure. In particular, these algebraic equations are most commonly rewritten as a boolean formula and solved using a SAT solver. Feasible SAT solving therefore translates directly to feasible cryptographic attacks.

Boolean Satisfiability (SAT) is the quintessential NP-complete problem and so all known algorithms are infeasible in the worst case. Despite this, the development of conflict-driven clause-learning (CDCL) solvers has recently led to a ``SAT Revolution" of high-performance solvers. \cite{BHMW09} Modern SAT solvers can realistically scale to many industrial formulas with millions of variables by combining backtracking search with rich heuristics. \cite{MZ09} SAT solvers have consistently improved from year to year, driven by a popular yearly competition among more than 60 solvers. \cite{JLRS12}

The rise of SAT solvers has enabled an entire host of cryptanalytic attacks. However, many of these attacks took place  \cite{CB07,CNO08,DKV07,MCP07,MZ06} eight or nine years ago. SAT solvers have continued to improve dramatically over this time on other applications. We hypothesize that these advancements in SAT solving will provide a notable decrease in the time taken to perform the attacks. We also hypothesize that, in cases where attacks only succeeded against a limited number of rounds, newer SAT solvers will succeed against more rounds. Overall, we expect newer SAT solvers to dramatically improve the feasibility of cryptographic attacks through SAT.

To test this hypothesis, in Section \ref{sec:encoding} we create SAT formulas that encode attacks on three distinct cryptographic encryption schemes and hash functions. In Section \ref{sec:results} we use these SAT formulas as benchmarks for various SAT solvers drawn from the past ten years.

Contrary to our hypothesis, we find that the improvements in SAT solver performance over the last ten years does not correspond to improvements in our cryptographic attacks. However, better hardware and better encodings of the cryptographic attacks \emph{do} allow for improved attacks using SAT solvers. We discuss this further and conclude in Section \ref{sec:conclusion}.
